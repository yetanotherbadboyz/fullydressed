\documentclass[12pt,a4paper]{report}
\usepackage{graphicx}
\usepackage[utf8x]{inputenc}
\usepackage[turkish]{babel}
\usepackage{fullpage}
\usepackage{multirow}
\usepackage{uml}

\begin{document}

\include{header}

\newpage
{
\bf
Supplementary Specification: \\[1cm]
}
\begin{tabular}{ |p{2.5cm} | p{2cm} | p{7cm} | p{2.7cm} | }

\hline
\bf
Version
&
\bf
Date
&
\bf
Description
&
\bf
Author \\
\hline

Inception Phase
&
17.03.2011
&
Elaboration fazında yeniden düzenlenmek üzere hazırlanmış ilk taslak.
&
Mesutcan Kurt
Orçun Avşar \\
\hline

&

&

&

&

\hline

\end{tabular}

\newpage

{
\bf
Glossary:  \\[1cm]
}
\begin{tabular}{ |p{2.5cm} | p{6cm} | p{2.3cm} | p{3cm} | p{2cm} |}
\hline
\bf
Term
&
\bf
Definition and Information
&
\bf
Format
&
\bf
Validation Rules
&
\bf
Aliases
&
\hline 
%Term
Müşteri
&
%Definition and Information
Otel kiralama/rezarvasyon işlemini resepsiyonist ya da 
turizm acentesi ile yapan, kiralanan oda ile kiralanan tarih
boyunca ilişkilendirilen otel müşterisidir.
&
%Format
&
%Validation Rules
&
%Aliases
&

\hline
Resepsiyonist
&
Resepsiyonist, bir otel personelidir. Otel ile ilgili olan rezervasyonları yapar. Oda kiralama işlemini gerçekleştirebilir. Sistemin de bir kullanıcısıdır.
&

&

&

&

\hline 
%Term
Otel
&
%Definition and Information
Sistemin tüm işlemlerinin gerçekleştiği fiziksel mekandır.
Otelin sahip olduğu yıldız sayısına göre bazı özellikleri müşterilerine
sunmak zorundadır.
&
%Format
&
%Validation Rules
&
%Aliases
&

\hline
Rezervasyon
&
Rezervasyon işlemi, belirli bir güne kadar, belirli bir gün için, belirli bir kişiye, belirli bir odayı ayırtma işlemidir.
&

&

&

&

\hline


\end{tabular}

\end{document}
