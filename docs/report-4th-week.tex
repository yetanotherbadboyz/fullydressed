\documentclass[12pt,a4paper]{report}
\usepackage{graphicx}
\usepackage[utf8x]{inputenc}
\usepackage[turkish]{babel}
\usepackage{fullpage}
\usepackage{multirow}

\begin{document}

\include{header}

\newpage

\begin{description}
\item[Projenin Konusu:] Otel Rezervasyon Sistemi \\[1cm]
\end{description}
{
\bf
Vizyon Amaçları
}

\begin{itemize}

\item {

Rezervasyon sistemi kullanmak isteyenlerin ihtiyacını karşılayacak bir yazılım sisteminin geliştirilmesi.

}

\item {

Sisteme otel kayıtlarının eklenmesi ve sistemi kullanacak kullanıcıların sisteme kaydedilmesi ve yetkilendirilmesi.

}

\item {

Sistemden faydalanacak otel müşterilerinin uygunluk durumuna göre istedikleri tarihlerde, istedikleri özelliklerde odalara yerleştirilmesi ve bunun için müşterilerden odanın özelliklerine, kaydı yapan yöneticilere (resepsiyonist, turizm acentesi vs.) ve diğer durumlara göre (kampanya, tatil günleri vs.) ücret talep edilmesi.

}

\item {

Müşterilere V.I.P gibi özelliklerin ilişkilendirilmesi ve müşterinin bunun sağladığı ayrıcalıklardan faydalanbilmesi. (Örn: Kral dairesininden sadece V.I.P. Müşteriler faydalanabilir)

}

\item {

Sisteme kaydedilen ve yetkilendirilen turizm acentaları tarafından da resepsiyonistin yapacağı işlemler yapılabilmelidir. Bu sayede otelin müşterileri fiziksel olarak otelde bulunmadan da rezarvasyon ve kiralama işlemlerini yapabileceklerdir. \\[2cm]

}

\end{itemize}

{
\bf
Projenin Taslağı \\[0.2cm]
}

Aşağıdakiler sistemin barındıracağı maddelerdir.

\begin{itemize}

\item Otel Ekleme/Değiştirme/Çıkartma
\item Oda Ekleme/Değiştirme/Çıkartma
\item Sistem Kullanıcıları Ekleme/Değiştirme/Çıkarma
\item Müşteri Ekleme/Değiştirme/Çıkarma
\item Oda Rezarvasyonu Yapma/İptal Etme
\item Oda Kiralama/İptal Etme

\end{itemize}

\newpage

{
\bf
\center
Actor-Goal Model \\[1cm]
}
\begin{tabular}{ |p{5cm} | p{9cm} |}

\hline
\bf Actor & \bf Goal \\ 
\hline

\multirow{2}{*}{Sistem Yöneticisi}
& 
Otel Ekleme/Değiştirme/Çıkartma \\
&
Otel Yöneticisi Ekleme/Değiştirme/Çıkartma \\

\hline

\multirow{3}{*}{Müşteri}
&
Kendisini Sisteme Ekleme/Değiştirme/Çıkartma \\
&
Rezervasyon Yapma/İptal \\
&
Oda Kiralama \\

\hline

\multirow{2}{*}{Turizm Acentası}
&
Oda(lar) Rezervasyonu Yapma/İptal \\
&
Oda(lar) Kiralama \\

\hline

\multirow{3}{*}{Resepsiyonist}
&
Müşteri Ekleme/Değiştirme/Çıkartma \\
&
Rezervasyon Yapma/İptal \\
&
Oda Kiralama/İptal \\

\hline

\multirow{2}{*}{Otel Yöneticisi}
&
Oda Ekleme/Değiştirme/Çıkartma \\
&
Otel Aktivitesi Ekleme/Değiştirme/Çıkartma \\

\hline

\end{tabular}

\newpage

\begin{description}
\item[Use Case 1:] Müşteri Ekleme \\
\item[Scope:] Otel Rezervasyon Sistemi
\item[Level:] User Goal
\item[Primary Actor:] Resepsiyonist 
\item[Stakeholders and Interests:] \hspace{10 mm}
\begin{description} 
\item[Resepsiyonist:] Müşterinin herhangi bir anahtar verisine göre 
(TC Kimlik No, OpenID, eposta) kaydını yapmak ister. Daha sonra bu 
müşteriye anahtar veri aracılığı ile ulaşmayı istemektedir.
\item[Müşteri:] Kaydının karışıklığa yol açmayacak şekilde tanımlanmasını istemektedir.
\end{description}
\item[Preconditions:] \hspace{10mm}
\begin{itemize}
\item Otel sistemde kayıtlı olmalı.
\item Resepsiyonist sistemde kayıtlı ve yetkiye sahip olmalı.
\item Resepsiyonist sisteme giriş yapmış olmalı.
\item Aynı ID'li başka bir müşteri eklenmemiş olmalı.
\end{itemize}

\item[Postconditions:] \hspace{10mm}
\begin{itemize}
\item Müşteri daha sonra erişebileceği anahtar verilerle eklenmiş olmalı.
\end{itemize}
\item[Main Success Scenario (or Basic Flow):] \hspace{10mm}
\begin{enumerate}
\item Müşteri otele gelir ve resepsiyoniste talebini iletir.
\item Resepsiyonist yeni bir kayıt işlemi başlatır.
\item Resepsiyonist müşteriden gerekli olan bilgileri temin eder.
\item Resepsiyonist verilen bilgilerin kanıtlanmasını talep eder.
\item Resepsiyonist bilgileri sisteme girer ve kayıt işlemini sisteme iletir.
\end{enumerate}
\item[Extensions (or Alternative Flows):] \hspace{10mm}
\begin{itemize}
\item[*a] Senaryodaki herhangi bir ölümcül hatada:
    \begin{enumerate}
    \item Bütün işlemler ipyal edilip her şey baştan yazılır.
    \end{enumerate}
\item[4a.] Müşteri kimliğini kanıtlayamaz ise
    \begin{enumerate} 
    \item Kayıt işlemi tamamlanmadan iptal edilir.
    \end{enumerate}
\end{itemize}

\end{description}

\newpage
\begin{description}
\item[Use Case 2:] Oda Kiralama \\
\item[Scope:] Otel Rezervasyon Sistemi
\item[Level:] User Goal
\item[Primary Actor:] Resepsiyonist 
\item[Stakeholders and Interests:] \hspace{10 mm}
\begin{description} 
\item[Resepsiyonist:] Oda kiralama işleminin her şeyiyle birlikte sorunsuz ve kolayca yapılabilmesini bekler
\item[Müşteri:] Oda kiralama işleminin yapılmasını bekler.
\item[Sistem Yöneticisi:] Oda kiralama işlemindeki bilgilerin eksiksiz olmasını bekler.
\end{description}
\item[Preconditions:] \hspace{10mm}
\begin{itemize}
\item Resepsiyonist sistemde kayıtlı ve yetkiye sahip olmalı.
\item Resepsiyonist sisteme giriş yapmış olmalı.
\item Sistemde en az bir otel olmalı.
\item Sistemde boş en az bir oda olmalı.
\item Müşteri sisteme kayıtlı olmalı.
\end{itemize}

\item[Postconditions:] \hspace{10mm}
\begin{itemize}
\item Oda kiralanmış ve müşteri ile ilişkilendirilmiş olmalı.
\end{itemize}
\item[Main Success Scenario (or Basic Flow):] \hspace{10mm}
\begin{enumerate}
\item Müşteri otele gelir ve resepsiyoniste talebini iletir.
\item Resepsiyonist yeni bir kayıt işlemi başlatır.
\item Resepsiyonist müşteriden gerekli olan bilgileri temin eder.
\item Resepsiyonist verilen bilgilerin kanıtlanmasını talep eder.
\item Resepsiyonist bilgileri sisteme girer ve kayıt işlemini sisteme iletir.
\end{enumerate}
\item[Extensions (or Alternative Flows):] \hspace{10mm}
\begin{itemize}
\item[*a] Senaryodaki herhangi bir ölümcül hatada:
    \begin{enumerate}
    \item Bütün işlemler ipyal edilip her şey baştan yazılır.
    \end{enumerate}
\item[3a] Müşteri sisteme kayıtlı değilse:
    \begin{enumerate}
    \item Yeni müşteri oluşturulup bu adımdan devam edilir.
    \end{enumerate}
\item[4a] Müşteriye uygun oda yoksa:
    \begin{enumerate} 
    \item Kayıt işlemi tamamlanmadan iptal edilir.
    \end{enumerate}
\item[6a] Ödeme yapılmazsa
    \begin{enumerate}
    \item İşlem iptal edilir.
    \end{enumerate}
\end{itemize}
\end{description}

\newpage
\begin{description}
\item[Use Case 3:] Oda Ekleme \\
\item[Scope:] Otel Rezervasyon Sistemi
\item[Level:] User Goal
\item[Primary Actor:] Sistem Yöneticisi 
\item[Stakeholders and Interests:] \hspace{10 mm}
\begin{description} 
\item[Sistem Yöneticisi:] Odanın tanımlanması ile ilgili herhangi bir eksiklik olmamasını ister.
\end{description}
\item[Preconditions:] \hspace{10mm}
\begin{itemize}
\item Otel sistemde kayıtlı olmalı.
\item Sistem yöneticisi sisteme kayıtlı ve giriş yapmış olmalı
\end{itemize}

\item[Postconditions:] \hspace{10mm}
\begin{itemize}
\item Oda rezervasyona hazır olmalı
\end{itemize}
\item[Main Success Scenario (or Basic Flow):] \hspace{10mm}
\begin{enumerate}
\item Sistem yöneticisi oda eklemek için yeni bir işlem başlatır.
\item Odanın numarasını ve diğer özelliklerini sisteme girer.
\item İşlem tamamlanır.
\end{enumerate}
\item[Extensions (or Alternative Flows):] \hspace{10mm}
\begin{itemize}
\item[*a] Senaryodaki herhangi bir ölümcül hatada:
    \begin{enumerate}
    \item Bütün işlemler ipyal edilip her şey baştan yazılır.
    \end{enumerate}
\item[2a] Aynı numaralı başka bir oda varsa:
    \begin{enumerate}
    \item Sistem yöneticiden başka bir oda numarası ister.
    \item Eşsiz bir oda numarası girilinceye kadar 2a tekrarlanır.
    \item İşlem kaldığı yerden devam eder.
    \end{enumerate}
\end{itemize}
\end{description}

\newpage
\begin{description}
\item[Use Case 4:] Oda Ekleme \\
\item[Scope:] Otel Rezervasyon Sistemi
\item[Level:] User Goal
\item[Primary Actor:] Sistem Yöneticisi 
\item[Stakeholders and Interests:] \hspace{10 mm}
\begin{description} 
\item[Sistem Yöneticisi:] Otelin özelliklerinin doğru tanımlanmış olmasını bekler.
\end{description}
\item[Preconditions:] \hspace{10mm}
\begin{itemize}
\item Sistem yöneticisi sisteme kayıtlı ve giriş yapmış olmalı.
\end{itemize}

\item[Postconditions:] \hspace{10mm}
\begin{itemize}
\item Otel sisteme tüm özellikleri ile kaydedilmiş olmalı.
\end{itemize}
\item[Main Success Scenario (or Basic Flow):] \hspace{10mm}
\begin{enumerate}
\item Sistem yöneticisi yeni bir otel kaydı başlatır.
\item Otel için gerekli bilgiler istenir.
\item Sistem yöneticisi gerekli bilgileri girer.
\item Kayıt işlemi tamamlanır.
\end{enumerate}
\item[Extensions (or Alternative Flows):] \hspace{10mm}
\begin{itemize}
\item[*a] Senaryodaki herhangi bir ölümcül hatada:
    \begin{enumerate}
    \item Bütün işlemler ipyal edilip her şey baştan yazılır.
    \end{enumerate}
\item[2a.] Aynı isimli başka bir otel varsa
    \begin{enumerate}
    \item Sistem yeni bir otel ismi girilmesini ister.
    \item Eşsiz bir otel ismi girilene kadar 2a devam eder.
    \item İşleme kalınan yerden devam edilir.
    \end{enumerate}
\item[2b.]
    \begin{enumerate}
    \item Sistem yeni bir adres girilesini ister.
    \item Eşsiz bir adres girilene kadar 2a devam eder.
    \item İşleme kalınan yerden devam edilir.
    \end{enumerate}
\end{itemize}
\end{description}

\newpage
\begin{description}
\item[Use Case 5:] Resepsiyonist Ekleme\\
\item[Scope:] Otel Rezervasyon Sistemi
\item[Level:] User Goal
\item[Primary Actor:] Sistem Yöneticisi 
\item[Stakeholders and Interests:] \hspace{10 mm}
\begin{description} 
\item[Sistem Yöneticisi:] Yeni bir resepsiyonistin sisteme kayıtlı olmasını bekler.
\end{description}
\item[Preconditions:] \hspace{10mm}
\begin{itemize}
\item Sistem yöneticisi sisteme kayıtlı ve giriş yapmış olmalı
\end{itemize}

\item[Postconditions:] \hspace{10mm}
\begin{itemize}
\item Resepsiyonist sisteme kaydedilmiş olmalı.
\end{itemize}
\item[Main Success Scenario (or Basic Flow):] \hspace{10mm}
\begin{enumerate}
\item Sistem yöneticisi yeni bir resepsiyonist kaydı başlatır.
\item Sistem resepsiyonist için gerekli bilgileri ister.
\item Sistem yöneticisi gerekli bilgileri girer.
\item Kayıt işlemi tamamlanır.
\end{enumerate}
\item[Extensions (or Alternative Flows):] \hspace{10mm}
\begin{itemize}
\item[*a] Senaryodaki herhangi bir ölümcül hatada:
    \begin{enumerate}
    \item Bütün işlemler ipyal edilip her şey baştan yazılır.
    \end{enumerate}
\item[2a] Aynı kullanıcı isimli başka bir resepsiyonist varsa
    \begin{enumerate}
    \item Sistem yeni bir kullanıcı adı girilmesini ister.
    \item Eşsiz bir isim girilene kadar 2a devam eder.
    \item İşlem kaldığı yerden devam eder.
    \end{enumerate}
\end{itemize}
\end{description}

\end{document}
